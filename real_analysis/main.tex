\documentclass{article}
\usepackage{graphicx} % Required for inserting images
\usepackage{amsmath, amssymb, amsfonts, amsthm} % Required for some math elements
\usepackage{hyperref}
\usepackage{enumitem}

\setlength{\parskip}{1em}
\setlength{\parindent}{0pt}
% Custom examlpe command
\newtheorem{example}{Example}[subsection]
\newtheorem{definition}{Definition}[subsection]
\newtheorem{remark}{Remark}[subsection]
\newtheorem{axiom}{Axiom}[section]
\newtheorem{proposition}{Proposition}[subsection]
\newtheorem{lemma}{Lemma}[subsection]
\newtheorem{theorem}{Theorem}[subsection]

\newcommand{\R}{\mathbb{R}}
\newcommand{\Q}{\mathbb{Q}}
\newcommand{\Z}{\mathbb{Z}}
\newcommand{\N}{\mathbb{N}}
\newcommand{\exercisesline}{	%Exercises line
    \begin{center}
    \textemdash\ Exercises\ \textemdash
    \end{center}
}
\newcommand{\mem}{\text{\textemdash}} % Math mode emdash
\newcommand{\vep}{\varepsilon} % Epsilon

% Special \pp operator for peano axioms
\newcommand{\pp}{\mathbin{+\mkern-5mu+}}

\let\bf\textbf
\let\it\textit
\let\tc\texttt

\title{Real Analysis Notes}
\author{Luke Bailey}
\date{}

\begin{document}

\maketitle

\noindent Notes from textbook \textit{Analysis I} by Terence Tao.

\tableofcontents
\newpage

\section{Introduction}

Real analysis is the study of real numbers. We will answer 
questions such as: What is a real number? How
do you take the limit of a squenece of real numbers? 
What is a continuous function? What is a derivative?

Analysis is important to study for both the 
satisfaction of knowing ``how'' real numbers work,
and to avoid falling into traps of false reasoning.
Consider: 

\begin{example}
	Consider $\lim_{x \to \infty} \sin(x)$. Make the 
	change of varialbe $y = x + \pi$ and recall 
	that $\sin(y + \pi) = - \sin(y)$. Then we get: 
	\begin{align*}
		\lim _{x \to \infty} \sin(x) &= \lim _{y + \pi \to \infty}
		\sin(y + \pi) 
		= \lim_{y \to \infty} - \sin(y) = - \lim_{y \to \infty}\sin(y)
	\end{align*}
	Naturally, $\lim_{x \to \infty}\sin(x)= \lim_{y \to \infty}\sin(y)$. Thus 
	we get: 
	\begin{align*}
		\lim_{x \to \infty}\sin(x) = - \lim_{x \to \infty}\sin(x)\\
		\lim_{x \to \infty}\sin(x) = 0
	\end{align*}
\end{example}

Another pitfall comes from interchanging sums. As we will see, 
whilst you can interchance finite sums, it is not always 
possible to interchange infinite sums. Similarly, 
we cannot always interchacne intergrals, or limits! What a bother. 

Another fun example is when the apply the famous L'Hopital's rule.
Written generically, this is:
$$
\lim_{x \to a} \frac{f(x)}{g(x)} = \lim_{x \to a} \frac{f'(x)}{g'(x)}
$$
But note that we can \emph{only} apply this rule if the limit 
of $x \to a$ of $f(x)$ and $g(x)$ are both zero. Even apart 
from this there are circumstances where the rule does not apply.

\newpage 

\section{The natural numbers}

We start with the natural numbers, $\mathbb{N} = \{0, 1, 2, 3, \ldots\}$.
Note that in this work, following Tao, we include zero in the natural numbers.
We start here because once we have $\mathbb{N}$ we can build 
$\mathbb{Z}$ and $\mathbb{Q}$, and then $\mathbb{R}$.

\subsection{The Peano axioms}

This is a standard way to define the natural numbers. This is not
the only way, and infact you can use the cardinality of sets 
to do it also.

\begin{definition}
	(Informal) The natural numbers are any element of the set: 
	$$
	\N = \{0, 1, 2, 3, \ldots\}
	$$
	which is the set of all numbers created by starting with 0 
	and counting forward indefinitely.
\end{definition}

This is an unsatisfactory definition, we don't know how to add
multiply etc.
What we can say is exponentiation is repeated multipliaction, 
which is repeated addition, which is repeated \it{incrementing}.

So do define natural numbers we will use two concepts: the zero number 
$0$, and the increment operation, which we denote with $\pp$. From this 
we say that $\N$ contains $0$ and everything that can be obtained 
by incrementing $0$.

\begin{axiom}
	(Zero is a natural number) $0 \in \N$	
\end{axiom}
\begin{axiom}
	If $n$ is a natural number, then $n\pp$ is also a natural number.
\end{axiom}

From this we see that $(0 \pp)\pp$ is a natural number, and so on. 

\begin{definition}
	We define $1$ to be the natural number $0\pp$, 2 to be 
	the number $(0\pp)\pp$, etc.
\end{definition}

Note that sets of modular numbers adhere to the previous two axioms.
We impose a third axiom to stop this.

\begin{axiom}
	0 is not the successor of any natural number.
\end{axiom}

\begin{proposition}
	$0 != 4$

	Proof. By axiom 2.1, $4 = 3\pp$. By axiom 2.3, $4 = 3\pp \neq 0$ 
	and thus we have $0 != 4$. \hfill$\blacksquare$
\end{proposition}

We still have problems though, take a number system 
with $1,2,3,4$ and $4\pp = 4$. This adheres to all of the axioms. 
We add a new axiom to account for this. 

\begin{axiom}
	For $n,m \in N$ such that $n \neq m$, then 
	$n\pp \neq m\pp$.
\end{axiom}

Finally we may have rogue elements of the number system. 
For example \\ $1, 2,\dots, a1, a2, \dots$ where $an+1 := an\pp$. 
We want an axiom that says the only numbers in $\N$ are those 
that are accessible by incrementing $0$. We do so using induxtion.

\begin{axiom}
	(Principle of mathematical induction). 
	Let $P(n)$ be a property pertaining to a natural number $n$.
	Suppose that $P(0)$ is true, and suppose that whenever $P(n)$ 
	is true, then $P(n\pp)$ is also true. Then $P(n)$ is true for
	all natural numbers $n$.
\end{axiom}

Note that the above axiom only holds (in particular the last statement 
that $P$ is true for all natural numbers) if all natrual numbers 
are ``reachable'' from $0$ by incrementing.

All of these axioms together are called the Peano axioms. We write 
them again here for completeness:

\begin{definition}
	(Peano axioms) The natural numbers are a set $\N$ together with
	a distinguished element $0 \in \N$ and a successor function $\pp: \N \to \N$
	which assigns to each natural number $n \in \N$ a natural number $n\pp \in \N$,
	such that the following axioms are satisfied:

	\begin{enumerate}
		\item (Zero is a natural number) $0 \in \N$
		\item (Successor function is defined on all of $\N$) For every $n \in \N$, $n\pp \in \N$.
		\item (Zero is not the successor of any natural number) For every $n \in \N$,
		\item (Successor function is injective) If $n,m \in \N$ and $n\pp = m\pp$,
			then $n = m$.
			we have $n\pp \neq 0$.
		\item (Principle of mathematical induction) Let $P(n)$ be a property 
			pertaining to a natural number $n$. Suppose that $P(0)$ is true, and 
			suppose that whenever $P(n)$ is true, then $P(n\pp)$ is also true. 
			Then $P(n)$ is true for all natural numbers $n$.
	\end{enumerate}
\end{definition}

\begin{proposition}
	(Recursive definitions). Suupose for each $n \in \N$ we have 
	a function $f_n : \N \to \N$, Let $c$ be a natural number, 
	then we can assign a unique natural numbers $a_n$ to each $n \in \N$
	such that $a_0 = c$ and $a_{n\pp} = f_n(a_n)$ for all $n \in \N$.


	Proof. We proove this by induction. We note that $a_0$ is unique 
	because $a_0 := c$, and we define $a_{n\pp} = f_n(a_n)$ for all $n \in \N$,
	and by Axiom 2.3 $0$ is the succesor of no natural number as thus 
	will never be redefined. Now let $a_n$ have a unique value. Consider 
	$a_{n\pp} := f_n(a_n)$. We have $a_{n\pp}$ will be unqiue with 
	value $f_n(a_n)$ as for all other $m \in \N$ we will have 
	$m\pp  \neq n\pp$ because of Axiom 2.4 (that is $\pp$ is injective).
	By induction we thus have this holds for all $a_n$.
\end{proposition}

What this says is that we can define a sequence of numbers recursively where 
each element of the sequence is well defined (that is it takes 
on  a unique value). Also note that we had to use all the axioms. In a 
number system that wraps around, a recursive definition would 
not provide unoiquness like this (which you can see trivially).

\subsection{Addition of natural numbers}

The idea from here is to build up more complex operations from increment.

\begin{definition}
	Let $m$ be a natural number. We define $0 + m := m$. Assume 
	that we have defined how to add $n$ and $m$. Then we 
	define $(n\pp) + m := (n + m)\pp$. By induction this will give 
	us the definition of all possible sums.
\end{definition}

This is similar to our previous discussion of recursive definitions.
In particular we have $a_n = n + m$ with $f_n(a_n) = a_n\pp$.

\begin{proposition}
	For any natural number $n$, $n + 0 = n$. 

	Proof. We use induction. Consider the base case $0 + 0 = 0$ by 
	the definition of addition. Now assume that $n + 0 = 0$. We now 
	show that $n\pp + 0 = n\pp$. By the definition of addition we have 
	$(n\pp) + 0 = (n + 0)\pp = n\pp$. \hfill $\blacksquare$
\end{proposition}


We now rattle off a few propoositions (mostly) without proof. In general they 
require using the above proposition, and the fact that 
$n + m\pp = (n + m)\pp$ (i.e., the symetric versions of the
properties given in the definition of additon) and induction
to prove.

\begin{proposition}
	(Various things about addition). For $a,b,c,d \in \N$ we claim
	(mostly) without proof:

	\begin{enumerate}
		\item  (commutative) $a + b = a +b$
		\item (associative) $(a + b) + c = a + (b + c)$
		\item (cancellation)  If $a + b = a + c$, then $b = c$.
	\end{enumerate}

	Proof of cancellation. As before we prove by induction on $a$. 
	Let $b$ and $c$ be arbitrary. Let $a = 0$. In this 
	case we have $0 + b = 0 + c \implies b =c $ by Proposition 
	2.3. Now assume that $a + b = a + c \implies b = c$ for 
	some other $a$. We now show the same holds for $a\pp$. 
	Let $(a\pp) + b = (a\pp) + c$. By the definition of addition
	we thus have $(a + b)\pp = (b +c)\pp$. As $\pp$ is injective 
	we must have $a + b = a + c$ and thus $b = c$. By induction,
	we complete the proof.
	\hfill $\blacksquare$
\end{proposition}

\begin{definition}
	(Positive natural numbers). $n \in \N$ is positive iff 
	it is not equal to 0.
\end{definition}

It quickly foloows that if $a$ is positive and $b$ is a natural 
number then $a +b$ is positive. We can prove this simply using
induction and the fact that $0$ is not the successor of any 
natural number (by Axiom of 2.3). A corollary of this is that 
if $a+b=0$, then $a=b=0$, else we would have a contradiction 
with the previous statement.

\begin{lemma}
	For $a\in\N$ that is positive there exists exactly one $b\in\N$ 
	such that $a = b\pp$. I.e. a is the successor of only 
	one natural number.
\end{lemma}

Having defined addition, we can now define a notion of 
order of the natural numebrs in terms of addition.

\begin{definition}
	(Ordering of natural numbers).	
	For $n,m \in \N$ we say that $n$ is less than or equal to $m$,
	written as $n \leq m$, iff there exists a natural number $a$ such that
	$n + a = m$. We say that $n$ is less than $m$, written as $n < m$,
	if $n \leq m$ and $n \neq m$.
\end{definition}

\begin{proposition}
	(Various properties of order). For $a,b,c \in \N$ we have:	

	\begin{enumerate}
		\item (Order is reflexive) $a \leq a$.	
		\item (Order is transitive) If $a \leq b$ and $b \leq c$, then $a \leq c$.
		\item (Order is anti-symmetric) If $a \leq b$ and $b \leq a$, then $a = b$.
		\item (Addition preserves order) If $a \leq b$, then $a + c \leq b + c$.
		\item $a < b$ iff $a\pp \leq b$.
		\item $a < b$ iff $b=a+d$ for some positive number $d$.
	\end{enumerate}
\end{proposition}

\begin{proposition}
	(Trichotomy of order). For any $a,b \in \N$ exactly one of the following 
	three statements is true: $a < b$, $a = b$, or $a > b$.	
\end{proposition}

Order allows us to create a \emph{stronger} principle of induction 
on the natural numbers.

\begin{proposition}
	(Strong principle of induction). Let $m_0$ be some natural 
	number, and let $P(m)$ be a property pertaining to an arbitrary
	natural number $m$. Suppose that for each $m > m_0$ we have 
	the following implication: if $P(m')$ is true for 
	all $m_0 \leq m' < m$, then $P(m)$ is also true. Then 
	we can conclude that $P(m$ is trual for all natural numbers $m \geq m_0$.
\end{proposition}

So what is this principle saying? Essentially, if $P$ is 
true for all $n \in [m_0, m)$ implies that $P(m)$ is true, then 
$P$ is true for all $n \geq m_0$. This is just like normal induction,
except in normal induction we have: 
$$
[P(0) \land P(n) \implies P(n\pp)] \implies P(n) \text{ is true for all } n \in \N
$$
Now we have: 
\begin{align*}
[P(m_0) \land P(m') \text{ is true for all } m_0 \leq m' < m \implies P(m)] 
\implies \\
P(n) \text{ is true for all } n \geq m_0
\end{align*}

We tend to normally use this with $m_0 = 0$ or $m_0 = 1$.


\exercisesline

\emph{Exercise 2.2.6}. Let $n$ be a natural number, and let $P(m)$ 
be a property pertaining to the natural numbers such that whenever
$P(m\pp)$ is true, then $P(m)$ is true. Suppose that $P(n)$ is true.
Prove that $P(m)$ is true for all natural numbers $m \leq n$.
This is known as the principle of backwards induction. (Hint: apply 
induction to the variable $n$).

\emph{Proof.} We induct on $n$. Consider the base case $n = 0$. In this 
case we know that $P(0)$ is true. There is no $m\in\N$ such that $m <0$, 
as this would imply that $0$ is the successor of some natural number,
which breaks one of the Peano axioms. Thus $P(m)$ is true for all
natural numbers $m \leq 0$. 

Now assume the claim is true for some $n$. We now show that it 
is also true for $n\pp$. We suppose that $P(n\pp)$ is true. As
everty natual number has a single successor, then $n$ is the single
predecessor of $n\pp$. By the induction hypothesis, $P(m)$ is true
for all $m \leq n$. As $P(n\pp$ is true by assumption, then $P(m)$
is true for all $m \leq n\pp$. By induction, the claim is true for
all $n \in \N$. \hfill $\blacksquare$

\subsection{Multiplication of natural numbers}

From here out we assume we can use all the properties of addition that 
we know without proof. We now move to multiplication, which is simply 
the iterated additation operation, in much the same way that addition
is simply iterated incrementation.

\begin{definition}
	Consider $m\in\N$. We define $0 \times m := 0$. Suppose 
	inductively that we have defined how to multiple $n$ and $m$. 
	Then we can multiple $n++$ to $m$ by defining $(n\pp)
	\times m := (n \times m) + m$. By induction this will give us
	the definition of all possible products.
\end{definition}

As before we can go ahead and prove lots of lemmas. Such as 
the distributive law which states that $a(b+c) = ab + ac$, 
commutativity, associativity, etc. Some useful ones are shown 
below.

\begin{proposition}
	(Useful facts about multiplication). We have:	

	\begin{enumerate}
		\item (Multiplcation preserves order). If $a,b \in N$ such  
		that $a < b$ and $c$ is positive, then $ac < bc$.
		\item (Cancellation). If $a,b,c \in \N$ and $a \neq 0$ such that
		$ab = ac$, then $b = c$.
	\end{enumerate}
\end{proposition}

Now here comes a big one! Very useful. 

\begin{proposition}
	(Euclidean algorithm). Let $n\in\N$ and $q$ be positive. Then there 
	exists natural numbers $m$, $r$ such that $0 \leq r < q$ and $n = mq + r$.
\end{proposition}

\begin{remark}
	In other words, we can divide a natural number $n$ by 
	a positive number $q$ to obtain a quotient $m$ and a remainder $r$.
	This was the beginning of \emph{number theory}.
\end{remark}

Going deeper, we can now define exponentiation in terms of multiplicaiton.

\begin{definition}
	(Exponentiation). We define $m^0 := 1$ for all $n \in \N$. Suppose	
	we for $n$ and $m$ we have defined $m^n$. Then we define 
	$m^{n\pp} := m^n \times m$. By induction this will give us the
	definition of all possible exponentiations.
\end{definition}
\newpage


%%%%%%%%%%%%%%%%%%%%%%%%%
% Chapter 3		%
%%%%%%%%%%%%%%%%%%%%%%%%%

\section{Set Theory}

Here we introduce some of the core aspects of axiomatic set theory, 
which almost every other branch of mathematics relies on. We leave 
discussion of more advanced topics, such as infinite sets, and 
the axiom of choice, to Chapter 8.

\subsection{Fundamentals}

Just like the natural numbers, we build what a set is using axioms. We start,
however, with an informal definition.

\begin{definition}
	(Sets, informal). We define a set A to be an unordered 
	colection of objects. If $x$ is an object, we say that $x$
	is an element of $A$, and write $x \in A$, if $x$ belongs to
	$A$. If $x$ is not an element of $A$, we write $x \notin A$.
\end{definition}

This definition is intuitive, but doesn't let us do things like 
set operations, or say what collections of objects are sets and what 
aren't. 

\begin{axiom}
	(Sets are objects). If A is a set, then A is also an object. Thus 
	given two sets, we can ask is one an element of the other.
\end{axiom}

\begin{remark}
	There is a special case of set theory called ``pure'' set
	theory in which \it{all} objects are sets. E.g. $0$ 
	is the empty set, and 1 is $\{0\} = \{ \emptyset \}$ and 
	so forth. From a logical point of view, pure set theory 
	is simpler as we only have to deal with one type of object. 
	From  a conceptual point of view, it is easier to deal 
	with impure set theory, as we are doing here.
\end{remark}

For us, of all the objects in maths, some are sets, and some are not. If
$x$ is an object and $A$ a set then $x \in A$ is true or false. 
If $A$ is not a set, then $x \in A$ is undefined. E.g., $3\in 4$ 
is niether true nor false, but meaningless.

\begin{definition}
	(Equality of sets). Two sets $A$ and $B$ are equal iff every 
	element of $A$ is in $B$ and vice versa.
\end{definition}

We informally introduce the axiom of substitution. This states 
that if two sets are equal, then we can replace one by the other 
in any expression, and the result will be unchanged. Note  that 
$\in$ respects this axiom, as if $x \in A$ and $A=B$ then $x \in B$. 
Thus if we build all set operations from $\in$, they too will obey 
the axiom. Futher note, however, that we do not care about the 
order of sets, so operations such as ``the first in'' and ``the last in''
\it{would not} obey the axiom of substitution.

We now define what objects are sets in a similar way to defining the natural 
numbers (where we started with $0$ and built up numbers from there).

\begin{axiom}
	(Empty set). There exists a set $\emptyset$, known as the empty 
	set, which contains no elements. That is $\forall x, x \not \in \emptyset$.
\end{axiom}

\begin{lemma}
	(Single choice). If $A$ is a non-empty set, then there exists 
	an object $x \in A$. 

	Proof. Assume towards contradiction this is not true. Then 
	$\forall x, x \not \in A$, thus $A = \emptyset$ which is a 
	contradiction.
\end{lemma}

This lemma is trivial, but says something fairly profound. We can 
always pick an elemnt from a non-empty set. Going further, given 
a finite set of non-empty sets, we can choose an element from each, 
known as ``finite choice.'' The extension of this to infinite sets 
reqiures another axiom, the \it{axiom of choice}, whic we discuss 
in Section 8.

\begin{axiom}
	(Singleton and pair sets). For every object $a$ there is 
	a singleton set $\{a\}$ whose only element is $a$, and for
	every pair of objects $a,b$ there is a pair set $\{a,b\}$
	whose only elements are $a$ and $b$.
\end{axiom}

Note that the singeton set axiom is redundent as it follows from 
the pair set axiom, as that gives us there exists sets 
$\{a,a\}$ which by our definition of set equivalence, is the same 
as $\{a\}$. The next axiom allows us to build bigger sets (and the 
it + the singleton axiom also gives us the pair set axiom).

\begin{axiom}
	(Pairwaise union). Given sets $A,B$, there exists a set 
	$A \cup B$ whose elements consist of all the elements 
	belonging to $A$ or $B$ or both. That is: 
	$$
	x \in A \cup B \iff (x \in A \lor x \in B)
	$$
\end{axiom}

By usion the definition of set equality and set union, you can 
see that $\cup$ obeys the axiom of substitution, and so is 
well defined on sets.

\begin{lemma}
	$\cup$ is commutative and associative. That is, for all sets	
	$A,B,C$ we have:
	\begin{enumerate}
		\item $A \cup B = B \cup A$
		\item $(A \cup B) \cup C = A \cup (B \cup C)$
	\end{enumerate}
	We leave the proof as an exercise to the reader.
\end{lemma}

Pairwise union lets us build sets with 2 objects, 3 objects etc. 
Note however, we cannot yes construct sets consisting of
$n$ objects for any $n\in\N$ as we have not regierously 
defined the concept of $n$-fold iteration. Similarly we cannot 
create infinite sets. We will introduce axioms later that 
allow us to do this.

\begin{definition}
	(Subsets). Given two sets $A,B$, we say that $A$ is a subset 
	of $B$, written $A \subseteq B$, iff every element of $A$ is 
	also an element of $B$. That is: 
	$$
	A \subseteq B \iff \forall x, x \in A \implies x \in B
	$$
	$A$ is a proper subset of $B$, written $A \subset B$, iff
	$A \subseteq B$ and $A \neq B$.
\end{definition}

As this definition only uses $\in$ and $=$, it obeys the axiom 
of substitution and thus is well defined.

\begin{proposition}
	(Sets are paritally ordered by set inclusion). For sets 
	$A,B,C$, if $A \subseteq B$ and $B \subseteq C$, then
	$A \subseteq C$. If $A \subseteq B$  and $B \subseteq A$, 
	then $A=B$. Finally, if $A \subset B$ and $B \subset C$ 
	then $A \subset C$.

	Proof. We leave the proof as an exercise to the reader.
\end{proposition}

\begin{remark}
	Note that sets are \it{partially} ordered by set inclusion. 
	This is because not all pairs of sets can be related using
	set inclusion. Thus the relation does not apply to all 
	pairs of sets. In contrast, $<$ as defined on natural 
	numbers is a \it{total} order, as it applies to all pairs
	of natural numbers.
\end{remark}

\begin{axiom}
	(Axiom of specification). Let $A$ be a set, and for each 
	object $x$ let $P(x)$ be a property pertaining to $x$. Then 
	there exists a set, denoted $\{x \in A : P(x) \}$, whose
	elements are exactly the elements $x$ in $A$ for which $P(x)$
	is true. In other words, for any object $y$:
	$$
	y \in \{x \in A : P(x) \} \iff (y \in A \land P(y))
	$$
	The axiom is also known as the axiom of separation.
\end{axiom}

As before, the axiom of specification does not break the axiom 
of substitution, and so is well defined. This axiom can be used 
to define other operations, such as intersections.

\begin{definition}
	(Set intersection). The intersection $S_1 \cap S_2$ we define as: 
	$$
	S_1 \cap S_2 := \{x \in S_1 : x \in S_2\}
	$$
\end{definition}

\begin{definition}
	(Set difference). We define $A - B$ or $A\setminus B$ as: 
	$$
	A - B := \{x \in A : x \notin B\} 
	$$
\end{definition}

\begin{proposition}
	(Sets form a boolean algebra). For sets $A,B,C$, all contained in $X$, 
	we have: 
	\begin{enumerate}
		\item (Minimal element) $A \cup \emptyset = A$ and $A \cap 
		\emptyset = \emptyset$.
		\item (Maximal element) $A \cup X = X$ and $A \cap X = A$.
		\item (Commutative laws) $A \cup B = B \cup A$ and $A \cap B = B \cap A$.
		\item (Associative laws) $(A \cup B) \cup C = A \cup (B \cup C)$ and $(A \cap B) \cap C = A \cap (B \cap C)$.
		\item (Distributive laws) $A \cap (B \cup C) = (A \cap B) \cup (A \cap C)$ and $A \cup (B \cap C) = (A \cup B) \cap (A \cup C)$.
		\item (Partition) $A \cup (X - A) = X$ and $A \cap (X - A) = \emptyset$.
		\item (De Morgan's laws) $X - (A \cup B) = (X - A) \cap (X - B)$ and $X - (A \cap B) = (X - A) \cup (X - B)$.
	\end{enumerate}
	These all follow fairly simply.
\end{proposition}


\begin{axiom}
	(Replacement). Let $A$ be a set. For any object $x \in A$, and 
	any object $y$, suppose we have a statement 
	$P(x,y)$ pertaining to $x$ and $y$, such that for each 
	$x \in A$, there is at most one $y$ for which $P(x,y)$ is true.
	Then there exists a set $\{y : P(x,y) \text{ is true for some } x \in A\}$, 
	such that for any object $z$:
	$$
	z \in \{y : P(x,y) \text{ is true for some } x \in A\} \iff P(x,z) \text{ is true for some } x \in A
	$$
\end{axiom}

\begin{example}
	Let $A := \{3,5,9\}$ and let $P(x,y)$ be $y = x\pp$. For every $x\in A$ 
	there is only one $y$ for which $P(x,y)$ is true as every number 
	has only 1 successor. The above axiom says the the replacement exists,
	and is the following set: $\{4,6,10\}$.
\end{example}

Using function notation, the axiom can be used to construct sets of the form: 
$$
\{y : y = f(x) \text{for some} x \in A\}
$$
Here we are using the fact that a function maps an element to a single 
element (something we will define in more detail later, so really 
using functions here is a bit circular). Note that that is the 
key feature of $P(x,y)$ in the axiom, $x$ is mapped to at most 1 $y$.
There is a subtle difference here though, we can have no such $y$ 
that satisfies $P(x,y)$ which doesn't quite mesh with our function 
definition.
We will more commonly write set constructions like this as:
$$
\{ f(x) : x \in A\}
$$
The axiom of replacement can be combined with the axiom of specification 
to create sets of the form:
$$
\{ f(x) : x \in A \land P(x)\}
$$

We now use sets to formalize natural numbers with our friends
the peano axioms.
\begin{axiom}
	(Infiniy). There exists a set $\N$, whose elemetns are called 
	natural numebrs, as well as an object $0$ in $\N$, and 
	an object $n \pp$ assigned to every $n \in \N$, such that 
	the Peano axioms hold.
\end{axiom}
This is the formal construction of the natural numbers! It is 
called the axiom of infinity because it introduces the basic example 
of an infinite set. The axiom of infinity gives us natural numbers, 
so gives us that numbers are objects in set theory.

\exercisesline

\it{Exercise 3.1.11}. Show that the axiom of replacement implies the
axiom of specification. 

\it{Proof (Informal)}. Here we do something like take a property 
$P(x)$ and define a replacement property $Q(x,y) := x = y \land P(x)$. 
Therefore when we apply the replacement, for any $x$ such that $P(x)$ is 
truen, $x$ is places in the resulting set, and if $P(x)$ is not true, 
then it is removed. Very nice!

\subsection{Russell's Paradox (Optional)}

We may be tempted to include the following axiom: 

\begin{axiom}
	(Universal specification). (Dangerous!) Suppose for every
	objet $x$ we have a property $P(x)$ pertaining to $x$ that 
	is true or false. Then there exists a set $\{x : P(x) \text{ is true}\}$
\end{axiom}

This axiom asserts that every property corresponds to a set, and thus that 
we can make sets of all of a thing. For example the set of all blue things, 
or the set of all sets. This axiom implies many of our previous axioms. 
Unfortunately it leads to a \bf{paradox} known as 
\it{Russell's Paradox}, and thus cannot be included in set theory.

The paradox is as follows. Define $P(x)$ as: 

$$
P(x) \iff x \text{ is a set, and } x \notin x 
$$

So $P(\{1,2\})$ is true, but if $S$ is the set of all sets, which 
we can construct from universal specification, then $P(S)$ is false.
Now consider, again through universal specification: 

$$
\Omega := \{x : P(x) \} = \{x : x \text{ is a set, and } x \notin x\}
$$

Now is $\Omega \in \Omega$? If it is, then $P(\Omega)$ is false and so it 
should not be. If it is not, then $P(\Omega)$ is true and so it should be. 
This is a contradiction, and so we cannot have the axiom of universal
specification.

The problem with the axiom is that it creates sets that are too big.
One way to resolve this is to put objects into heirarchies. At the bottom 
are primitive objects which are not sets. One layer up there 
are sets of primitive objects. Then above this there are sets 
that contain primitive objects and sets of primitive objects, etc. 
This means sets at each stage of the heirarchy can only contain 
things from lower stages, and thus \it{a set can never contain itself}.
Formalizing this is difficult, instead we include an axiom that means 
we do not run into Russell's paradox.

\begin{axiom}
	(Regularity or foundation). If $A$ is a non-empty set, then
	there is at least one element of $A$ which is either not a set, 
	or is disjoint from $A$
\end{axiom}

This axiom implies that sets cannot contain themselves. For the purpose of 
doing analysis, this axiom is never needed, so should be considered a side point.

\exercisesline

\it{Exercise 3.2.2}. Use the axiom of regularity and singleton set axiom to 
show that if $A$ is a set, then $A \not \in A$.

\it{Proof}. Assume towards contradiction $A \in A$. By the singleton 
set  axion we can make $B := \{A\}$. Not that $B$ breaks the axiom 
of regularity, as it has 1 element that is a set, and that one element $A$ 
is not disjoint from $B$, as $B \cap A = A \neq \emptyset$ as $A \in A$. Thus 
we have a contradiction, and so $A \not \in A$. \hfill $\blacksquare$


\subsection{Functions}

We start with a fareful cumbersome, but precise definition of a function.

\begin{definition}
	(Functions). Let $X,Y$ be sets, and let $P(x,y)$ be 
	a property pertaining to an object $x \in X$ and $y \in Y$, such
	that for every $x \in X$ there is exactly one $y \in Y$ for which 
	$P(x,y)$ is true \footnote{This is sometimes known as the vertical
	line test}. We define the function $f: X \to Y$ defined by $P$ 
	on the domain $X$ with range $Y$ to be the object which, 
	given any input $x \in X$, assings an output $f(x) \in Y$, 
	defined to be the unique object $f(x)$ for which $P(x, f(x))$ os 
	true. Thus, for any $x \in X$ and $y \in Y$, we have: 

	$$
	y = f(x) \iff P(x,y) \text{ is true}
	$$
\end{definition}

Functions are also referred to as \it{maps} and \it{transformations}.
depending on the context. They are also sometimes called 
\it{morphisms}, although to be more precise, a morphism refers to a
more general class of object, which may or may not be a function.

There are two ways to define functions: 
\begin{enumerate}
	\item Explicitly. In this case say what $x$ gets mapped to, for example 
		$f(x) = x\pp$. 
	\item Implicitely. In this case we define a property $P(x,y)$ that 
		links $x$ with the output of $f(x)$. When doing this we 
		need to be careful to ensure that the property is well defined 
		(that is there is a single $y$ for each $x$ such that 
		$P(x,y)$ is true).
\end{enumerate}

Functions obey the axiom of substitution, that is $x = x' \implies f(x) = f(x')$,
because the property $P(x,y)$ obeys the axiom. We define some useful 
properties of functions: 

\begin{definition}
	(Lots of things about functions). We have: 
	\begin{enumerate}
	\item Equality. For $f$ and $g$ with saem domain and range, $f = g$ iff
		$\forall x, f(x) = g(x)$.
	\item Composition. For $f:X\to Y$ and $g:Y\to Z$, the composition 
		$g \circ f: X \to Z$ is defined explicitely as: 
		$$
		(g \circ f)(x) := g(f(x))
		$$
		If the range of $f$ is not the domain of $g$, the operation 
		is undefined. This obeys the axiom of substitution. Composition 
		is associative ($f \circ (g \circ h) = (f \circ g) \circ h$).
	\item Injective (one-to-one). A function $f:X\to Y$ is injective iff 
		$f(x) = f(x') \implies x = x'$. Taking the contrapositive,
		this means that $x \neq x' \implies f(x) \neq f(x')$.
	\item Surjective (onto). A function $f:X\to Y$ is surjective iff 
		$\forall y \in Y, \exists x \in X$ such that $f(x) = y$.
	\item Bijective. If $f$ is surjective and injective, then it is 
		bijective.
	\end{enumerate}
\end{definition}

\subsection{Images and Inverse Images}

\begin{definition}
	(Images of sets). Consider $f: X \to Y$ and $S \subseteq Y$, 
	we define $f(S)$ as
	$$
	f(S) := \{f(x) : x \in S\} \subseteq Y
	$$
	and this is called the image of $S$ under $f$. Sometimes 
	$f(S)$ is called the forward image.
\end{definition}

Note that the image is well defined using the axiom of replacement.
You can also define $f(S)$ using the axiom of specification (specifying 
elements of $Y$ to include as opposed to elements of $X$ to replace).

\begin{definition}
	(Inverse image). If $U \subseteq Y$, we define $f^{-1}(U)$ to be:
	$$
	f^{-1}(U) := \{x \in X : f(x) \in U\} \subseteq X
	$$
\end{definition}

We note that functions are objects, and so we should be able to consider 
sets of \it{all} functions from a set $X$ to a set $Y$. Remember 
from Russell's paradox that we cannot make this using the axiom 
of universal specification, as we did not introduce this into 
set theory (due to the paradox it creates). Therefore we introduce 
a new axiom for this specifically. 

\begin{axiom}
	(Power set axiom). Let $X$ and $Y$ be sets. There exists a set, 
	denoted by $Y^X$, which consists of all the functions from 
	$X$ to $Y$, thus: 
	$$
	f \in Y^X \iff (f \text{ is a function with domain } X 
	\text{ and range } Y)
	$$
\end{axiom}

The reason we use the notation $Y^X$ is if $Y$ has $n$ elements and $X$ has 
$m$ elements, then $Y^X$ has $n^m$ elements.

\begin{lemma}
	Let $X$ be a set. Then the following is a set 
	$$
	\{Y : Y \subseteq X \}.
	$$
	We refer to this as the pwer set of $X$, denotes by 
	$2^X$ or $\mathcal{P}(X)$.
\end{lemma}

To construct this set, we use the Power set axiom to get $X^X$, and then
use the axiom of specification to select all the distinct subsets of $X$, 
as for each $A \subseteq X$, there is a function $f: X \to A$ that is 
in $X^X$.

Finally we enhance the axiom of pairwise union to create much larger sets.

\begin{axiom}
	(Union). Let $A$ be a set, all of whose elements are sets themselves.
	Then there exists a set $\bigcup A$ whose elements 
	are precisely those objects that are elements of the elements of A, thus 
	for all objects $x$ we have: 
	$$
	x \in \bigcup A \iff \exists B \in A, x \in B
	$$
\end{axiom}

The axiom of union with the axiom of pair set, implies the axiom of pairwise
union. Another important consequence is if we hvae an set $I$, and for 
all $\alpha \in I$ we have some $A_\alpha$, then we can form the union 
set $\bigcup_{\alpha \in I} A_\alpha$ by defining: 
$$
\bigcup_{\alpha \in I} A_\alpha := \bigcup \{A_\alpha : \alpha \in I\}
$$
where $\{ A_\alpha : \alpha \in I\}$ is a set by the axiom of replacement. 
We often call such an $I$ an indexing set. We can also form intersections 
like this as: 
$$
\bigcap_{\alpha \in I} A_\alpha := \{x \in A_\beta : x\in A_\alpha \forall \alpha \in I\}
$$
where $I$ here is non-empty and $\beta \in I$ is arbitrary. Note this 
is a set by the axiom of specification (we are simply specifying what 
elements of $A_\beta$ to include).

The axioms of set theory that we have introduced here, excluding the 
dangerous axiom of universal specification, are known as the
\it{Zermelo-Fraenkel axioms of set theory}. The axioms of
There is one further axiom, the \it{axiom of choice}, which 
gives rise to the \it{Zermelo-Fraenkel-Choice axioms of set theory},
but we won't need it until much later. We recap the axioms now: 

\begin{definition}
	(Zermelo-Fraenkel axioms of set theory). We have:	
	\begin{enumerate}
		\item (Empty set). There exists a set $\emptyset$.
		\item (Set equality). Two sets are equal iff they have the same elements.
		\item (Pair set). For any $a,b$ there exists a set $\{a,b\}$.
		\item (Union). For any set $A$ there exists a set $\bigcup A$.
		\item (Power set). For any sets $X$ and $Y$ there exists a set $Y^X$.
		\item (Axiom of specification). For any set $A$ and property $P(x)$, 
			there exists a set $\{x \in A : P(x)\}$.
		\item (Replacement). For any set $A$ and property $P(x,y)$, 
			there exists a set $\{y : P(x,y) \text{ is true for some } x \in A\}$.
		\item (Infinity). There exists a set $\N$.
		\item (Regularity). If $A$ is a non-empty set, then there is 
			at least one element of $A$ which is either not a set, 
			or is disjoint from $A$.	
	\end{enumerate}
\end{definition}

NOTE: \it{Technically} Zermelo-Fraenkel set theory is a form of pure set 
theory, which changes some things. Most notably the axiom of infinity 
does not use the natural numbers (as in pure set thoery, there 
are only sets), and instead creates an infinite set of sets that 
escalate up the heirarchy we described earlier. The minimal set
that satisfies this axiom, known as the \it{von Neumann ordinal}, is
equivalent to the natural numbers. For our purposes, we study 
non-pure set theory, so we define things slightly differently. Additionally, 
the Power set axiom refers to $2^X$ not $Y^X$. Finally,
Zermelo-Fraekel set theory only has 2-9 of the above axioms 
(with some changes like the afformentioned one), and the axiom of
the empty set arrises from the axiom of infinity. 


\subsection{Certesian Products}

This is another fundamental operation on sets. 

\begin{definition}
	(Ordered pairs). If $x$ and $y$ are any object, we define the 
	ordered pair $(x,y)$ to be a new object, consisting of $x$ 
	as its first component and $y$ as its second component. Two 
	ordered pairs $(x,y)$ and $(x',y')$ are equal iff $x=x'$ and $y=y'$.
\end{definition}

Technically this is partly an axiom, because we have postulated that given any 
two objects $x$ and $y$, the object $(x,y)$ exists. It is, howeer, 
possible to define an ordered pair using the axioms of set theory in such 
a way that we do not need any further axioms (see Exercise 3.5.1). 

\begin{definition}
	If $X$ and $Y$ are sets, then the cartesian product
	$X \times Y$ is defined as: 
	$$
	X \times Y = \{(x,y) : x \in X \land y \in Y\}
	$$
	or equivalently:
	$$
	a \in (X \times Y) \iff a = (x,y) \land x\in X \land  y \in Y
	$$
\end{definition}

\begin{definition}
	(Ordered $n$-tuple and $n$-fold cartesian product). An 
	oredered n-tuple $(x_i)_{1 \leq i \leq n}$, also 
	denoted as $(x_1, x_2, \dots, x_n)$ is a collection of 
	$n$ objects, where $x_i$ is the $i$th component. Two 
	ordered $N$-tuples are equal iff each of their components
	are equal, that is $x_i = y_i$. 

	If $(X_i)_{1 \leq i \leq n}$
	is an ordered n-tuple of sets, then their cartesian 
	product $\prod X_i$ is defined as
	$$
	\prod X_i := \{(x_i)_{1 \leq i \leq n} : x_i \in X_i \text{ for all } 
	1 \leq i \leq n\}
	$$
	This means take the first component from the first set, the
	second from the second set, etc, and make all the ones possible.
\end{definition}

This definition simply postulates that an ordered n-tuple and Cartesian
product always exist when needed, but using the 
axioms of set theiry once cna explicitly construct these objects.

\begin{remark}
	To get Cartesian product, we just need $n$-fold version. 
	All wee need to do is show $\prod X_i$ is a set. Here's 
	how we do it. With the power set axiom, we consider the set 
	of functions $i \mapsto x_i$ from $\{ 1 \leq i \leq n\}$ to 
	$\bigcup X_i$, that is the set $(\bigcup X_i)^{\{ 1 \leq i \leq n\}}$.
	Another way of writing this is the set: 
	$$
	(\bigcup X_i)^{\{ 1 \leq i \leq n\}} = \{ f\mid f: [n] \to \bigcup X_i\}
	$$
	Then we can use the axiom of specification to restrict this to 
	the set of functions tha map $i \mapsto x_i$ for $x_i \in X_i$. 
	That is the functions that when given a natural number $i$, select 
	and element from the set $X_i$. Note that each such function $f_j$ defines
	an $n$ tuple:
	$$
	(f_j(1), f_j(2), \dots, f_j(n)) \in \prod X_i
	$$
	With some more work we can show that there is a bijection between 
	the functions and all possible $n$ tuples in $\prod X_i$, and thus
	using the axiom of replacement, we can construct $\prod X_i$ is a set.
\end{remark}

An ordered $n$-tuple of objects is also called an ordered sequence of
$n$ elements, or a finite sequence.

\begin{lemma}
	(Finite choice). Consider $n \geq 1$ with $n \in \N$ and for all 
	$1 \leq i \leq n$ let $X_i$ be a non-empty set. Then there exists
	an $n$-tuple $(x_i)_{1 \leq i \leq n}$ such that $x_i \in X_i$ for
	$\forall i$. In other words, if each $X_i$ is non-empty, then 
	the set $\prod X_i$ is also non-empty.

	\it{Proof} (informal). We prove with induction. When $n=1$, we
	have the claim is true thanks to the \it{single choice lemma}. 
	Assume true for $n$. Consider $n +1$. In this case make the 
	$n+1$ tuple $(x_i)_{1 \leq i \leq n+1}$ by taking the $n$ tuple
	$(x_i)_{1 \leq i \leq n}$ and then using the single choice lemma
	to select $x_{n+1} \in X_{n+1}$. Thus the claim is true for
	$n+1$ and so by induction is true for all $n$. \hfill $\blacksquare$
\end{lemma}

This lemma is essentially trivial (although requires rigour to prove). 
It just said given a finite number of non-empty sets, we can pick
an element from each. It cannot, however, be extended to 
an infinite number of sets. For this we need the axiom of choice, which 
we will introduce later.
This is becuase we cannot induct infinitely.


\exercisesline

\it{Exercise 3.5.1}. Define an ordered pair $(x,y)$ to be the set
$\{\{x\}, \{x,y\}\}$. Show that this definition obeys the definition 
of an ordered pair.

\it{Proof}. Recall the definition of an ordered pair was simply:
$$
(x,y) = (x',y') \iff x=x' \land y=y'
$$
Consider two sets $A = \{ \{x\}, \{x,y\}\}$ and $B = \{ \{x'\}, \{x',y'\}\}$.
We prove both directions in turn. Assume that $\{ \{x\}, \{x,y\}\} = 
\{ \{x'\}, \{x',y'\}\}$. Thus the sets contain the same elements (by the 
Axiom of extensionality). Thus $\{ x\} \in B$, but $B$ contains only one 
element that is a set of one element, thus we must have $\{x\} = \{x'\}$.
Similarly we have $\{x,y\} \in B$, and so $\{x,y\} = \{x',y'\}$. Note
this implies $x = x'$ and $y = y'$, and thus.
$$
A = B \implies \{x\} = \{x'\} \implies  \{x,y\} = \{x',y'\} \implies x=x' \land y=y'
$$
The other direction is simpler, it is trivial that if $x = x'$ and $y = y'$ 
then $A = B$. Thus we have: 
$$
A = B \iff x=x' \land y=y'
$$
as required. \hfill $\blacksquare$

\it{Exercise 3.5.12}. TODO This exercise looks really interesting.

\subsection{Cardinality of sets} In the previous chapter we defined 
the natural numbers axiomatically. We: 
\begin{itemize}
	\item Assumed there was $0$ and an increment operation.
	\item Assumed $5$ axioms of how these interact.
\end{itemize}
This is philosophically different to thinking of numbers as 
``how many things there are,'' or more formally, the cardinality
of sets.

The Peano axiom approach treats numbers as more \it{ordinals}
than cardinals. Cardinals are One, Two, Three, etc., and are 
used to count how many things there are in a set. Ordinals 
are First, Seond, Third, etc., and are used to order a sequence 
of objects. \it{There is a difference between the two}, which 
arises when considering infinite ordinals and infinite cardinals, 
although we don't need to worry about this.

In the previous section chapter we did not answer the question 
``can natural numbers be used to \it{count} sets.'' Here 
we show that they can be used to count the cardinality of sets 
asssuming the sets are finite.

To start getting at this, we may want to answer a simpler question.
Not how many elements does a set have, but when do two 
sets have the same size. One way to do this 
is to say they have the same size when they have the same
number of elements, but this becomes circular as we have 
not defined ``number of elements'' and breaks down if 
we consider infinite sets.

\begin{definition}
	(Equal cardinality). Two sets $X$ and $Y$ have equal cardinaloty 
	iff there exists a bijections $f: X \to Y$. 
\end{definition}

Note, interestingly, that we don't know yet if $\{1,2\}$ and $\{1\}$ are 
not the same cardinality? One way to do this would be to enumerate all functions 
between them and show none are bijective. Weirdly, a set can contain 
another set as a proper subset and still have the same cardinality (only 
infinite sets), for example the even numbers and the natural numbers.

\begin{proposition}
	Equal cardinality is an equivalence relation. Recall this means 
	that the relation is: 
	\begin{itemize}
		\item Reflexive: $X$ has the same cardinality as $X$.
		\item Symmetric: If $X$ has the same cardinality as $Y$, then 
			$Y$ has the same cardinal	
		\item Transitive: If $X$ has the same cardinality as $Y$, and 
			$Y$ has the same cardinality as $Z$, then $X$ has the same 
			cardinality as $Z$.
	\end{itemize}
\end{proposition}

Consider a natural number $n$. We now want to define what it \it{means} 
for a set to have $n$ elements.

\begin{definition}
	Let $n$ be a natural number. A set $X$ is said to have $n$ 
	elements if it has the same cardinality as the set 
	$\{1,2,\dots,n\}$. We also say that $X$ has cardinality $n$ 
	iff it has $n$ elements.
\end{definition}

Now lets make sure our definition does not lead to any craziness, such as
a set having two different cardinalities.

\begin{proposition}
	(Uniqueness of cardinality). If $X$ has cardinality $n$, then 
	it cannot have another cardinality $m \neq n$.

	\it{Proof} (informal). You start with a lemma that a set 
		with positive cardinality is non-empty, and if $x \in X$ 
		then $X - \{x\}$ has cardinlaity $|X| - 1$ (which here
		denotes the unique predecessor of $n$, as we have not
		defined negation yet). With this lemma, you prove the 
		proposition by inducting on $n$ the cardinality 
		of $X$. Consider we have the inductive assumption 
		and are looking at a set with cardinality $n\pp$ but
		also cardinality $m$ with $m \neq n$. Then we have
		$X - \{x\}$ has cardinality $n$ and $m-1$, and by the 
		inductive assumption $n=m-1$. But the Peano 
		axioms say each number has a unique successor, so $n\pp = m$
		which is a contradiciton.
\end{proposition}

Now we have defined cardinality and with this proposition, 
we know that $\{1,2\}$ and $\{1\}$ do not have the same cardinality 
and we \it{do not} have to enumerate all fnuctions between them 
and show none are bijective. Instead we just need to show 
a function for each to get the cardinality of one as $2$ 
and the other as $1$.

\begin{definition}
	(Finite sets). A set is finite iff it has a 
	cardinality $n$ for some natural number $n$.
	Otherwise the set is called infinite. If X is
	a finite set, we use $|X|$ to denote its 
	cardinality.
\end{definition}

\begin{theorem}
	The set of natural numbers $\N$ is ifinite.	

	\it{Proof}. ATC that this is false. Thus $\exists n \in \N$ 
	such that $|\N| = n$. Thus there $\exists$ a bijection 
	$f: \{1,2,\dots,n\} \to \N$. Consider $f(1), f(2), \dots, f(n)$. 
	We can show that this is bounded, that is there exists some $M > f(i)
	\forall i \in \{1,2,\dots,n\}$. Then consider $M+1$. This 
	is not mapped to by $f$, thus $f$ is not surjective and so 
	is not a bijection, which is a contradiction.
\end{theorem}

Now we switch gears for a second. Lets see if we can define airhtmetic 
in terms of the cardinality of sets as opposed to using the 
Peano axioms.

\begin{proposition}
	(Cardinal arithmetic)
	\begin{enumerate}[label=(\alph*)]
		\item Let $X$ be finite set and $x \not \in X$, then
			$|X \cup \{x\}| = |X| + 1$.
		\item If $X$ and $Y$ are finite sets then $X \cup Y$ 
			is finite and $|X \cup Y| \leq |X| + |Y|$. 
			If $X$ and $Y$ are disjoint then $|X \cup Y| = |X| + |Y|$.
		\item If $X$ is finite and $Y \subseteq X$ then $Y$ is finite
			and $|Y| \leq |X|$. If $Y \subset X$ then $|Y| < |X|$.
		\item If $X$ if finite and $f : X \to Y$ then $f(X)$ is finite 
			and $|f(X)| \leq |X|$. If $f$ is injective then
			$|f(X)| = |X|$.
		\item If $X$ and $Y$ are finite sets then $Y^X$ is 
			finite and $|Y^X| = |Y|^{|X|}$.
		\item If $X$ and $Y$ are finite sets then $X \times Y$ is finite
			and $|X \times Y| = |X| \times |Y|$.
	\end{enumerate}
\end{proposition}

The above propositions form the basis of arithmetic of natuaral numbers 
\it{without using the recursive Peano axioms}. This is the basis 
of \it{cardinal arithmetic}. For this work, we won't develop this 
arithmetic further and instead use the Peano axioms.

\exercisesline

\it{Exercise 3.6.10}. Let $A_1,\dots,A_n$ be finite sets such 
that $|\bigcup_{i=1}^n A_i|  > n$. Show thatb there exists 
$i \in \{1,...,n\}$  such that $|A_i| \geq 2$. This is known as 
\it{the pigeonhole principle}.

\it{Proof}. We prove by induction. For $n=1$ we have $|A_1| > 1$ 
and thus the claim holds. Assume true for $n \in \N$. Now consider 
$n +1$. We have $|\bigcup_{i=1}^{n+1} A_i| > n+1$. We have two cases: 
\begin{enumerate}
	\item $|A_{n+1}| \geq 2$ in which case we are done.
	\item $|A_{n+1}| = 1$. Thus $A = \{x\}$. Now remove this from 
		the union. Now if $\{x\} \not \in \bigcup_{i=1}^n A_i$ then 
		we have $|\bigcup_{i=1}^n A_i| > n$ and so by the inductive
		assumption we have $|A_i| \geq 2$ for some $i$. If
		$\{x\} \in \bigcup_{i=1}^n A_i$ then we have the size
		of the union is unchanged, so $|\bigcup_{i=1}^n A_i| > n$, 
		so again by the inductive assumption we have $|A_i| \geq 2$
		for some $i$.
\end{enumerate}

\newpage

%%%%%%%%%%%%%%%%%%%%%%%%%
% Chapter 4             %
%%%%%%%%%%%%%%%%%%%%%%%%%

\section{Integers and Rationals}

\subsection{Integers}

We want to introduce a notion of subtraction, on top of addition 
and multiplicaiton which we already have. Informally, the 
integers are what we get when subtracting two natural numbers.
This is not a complete definition because: 
\begin{enumerate}[label=\arabic*)]
	\item It doesn't say when two differences are equal.
	\item It doesn'y say how to add and multiply integers 
		(do arithmetic).
	\item Its circular, we haven't defined subtrafction yet, 
		and infact need the integers to do this.
\end{enumerate}
We will build the integers by defininging them to follow the algebraic
rules we know. For a), if $a-b = c-d$, then this means $a + d = b + c$.
So equality can be defined using addition. To answer b), we know 
that $(a - b) + (c -d) = (a + c) - (b + d)$ and $(a - b)(c - d) = ac - bd$,
so we can define addition and multiplication using these rules. 
Finally for c), we will being by writing integers as $a\mem b$ instead,
where $\mem$ is simply a placeholder symbol. Later when we define 
subtraction, we will see that $a-b = a\mem b$, and we can remove it. 

\begin{definition}
	(Integers). An integer is an expression of the form 
	$a\mem b$, where $a$ and $b$ are natural numbers. Two 
	integers are qual, $a\mem b = c\mem d$, iff $a + d = b + c$.
	We let $\Z$ denote the set of all integers.
\end{definition}

\begin{remark}
	This is not the most formal set theoretic definition. What 
	is an ``expression''? In the language of set theory, we 
	are imposing an equivalence relation $\sim$ on the space
	of $\N \times \N$ ordered pairs of natural number,
	where:
	$$
	(a,b) \sim (c,d) \iff a + d = b + c
	$$
	Following this, the set theoretic definition of $a \mem b$ is 
	the equivalence class of $(a,b)$:
	$$
	a \mem b := \{ (c,d) \in \N\times\N : (a,b) \sim (c,d) \}
	$$
	From this, we can us the normal definition of set equality
	to say $a \mem b = c \mem d$. 
	This interpretation plays no role in how we end up manipulating 
	integers, and infact thinking of integers as a set of 
	equivalent pairs of natural numbers is quite cumbersome.
\end{remark}

To check that this is a legitimate notion of equality, we need 
to make sure that it is reflexive, symmetric, and transitive, and 
obeys the substitution property. Note that we cannot verify the 
substiution axiom because we have not defined any binary operations
on the integers yet, luckily we only need to do it for the basic
operations as more complex operations will be built from these.

\begin{definition}
	The asume of two integers is defined as: 
	$$
	a \mem b + c \mem d := (a + c) \mem (b + d)
	$$
	The producwe of two integers is defined as: 
	$$
	(a\mem b)(c\mem d) := (ac + bd) \mem (ad + bc)
	$$
\end{definition}

\begin{lemma}
	(Addition and multiplication are well defined). We check 
	the axiom of substitution. 

	\it{Proof}. For brevity we just show addition. Consider 
	$a \mem b = a' \mem b'$. We want to show:
	$$
	(a \mem b) + (c \mem d) = (a' \mem b') + (c \mem d)
	$$
	We have: 
	\begin{align*}
		LHS &= (a+c)\mem (b + d) \\	
		RHS &= (a' + c) \mem (b' + d)
	\end{align*}
	Now recall the definition of equality on the integers. We have 
	$LHS = RHS  \iff a+c + b'+d = a' + c + b + d$. Note that we have
	$$
	a - b = a' -b' \implies a + b' = a' + b
	$$
	By adding $c+d$ to both sides we get:
	$$
	a + c + b' +d = a' + c + b +d \implies LHS = RHS
	$$
	which concludes the proof. \hfill $\blacksquare$
\end{lemma}

The integers $n\mem 0$ behave the same way as the natural numbers
(we can show that addition and multiplication works the saem way,
and that $n\mem 0 = m \mem 0 \\ \iff n = m$). We say there is 
an \it{isomorphism} between $\N$ and the integers of the form $n\mem 0$.
This allows us to identify the natural numbers with integers by 
setting $n = n\mem 0$. We can now define incrementation on the integers 
by defining $x \pp = x+1$. 

\begin{definition}
	(Negation of integers). If $(a\mem b)$ is an integer, we define
	the negation $-(a\mem b)$ to be the integer $(b\mem a)$.
	In particular if $n = n \mem 0$ is a positive natural number, 
	we define the negation $-n = n\mem 0$. (We leave showing 
	that this definition is well defined as an exercise for the 
	reader).
\end{definition}

\begin{lemma}
	(Trichotomy of integers). Let $x$ be an integer. Then exactly	
	Then exactly one of the following statements is true:
	\begin{enumerate}
		\item $x$ is 0.
		\item $x$ is equal to a positive natural numebr $n$,
		\item $x$ is the negation $-n$ of a positive natural 
			number $n$.
	\end{enumerate}

	\it{Proof}. We first show that at least one of the above is true.
	By definition $x = a\mem b$ for $a,b \in \N$. By the trichotomy
	of natural numbers, we have $a = b$, $a < b$ or $a > b$. We consider 
	each case in turn:
	\begin{itemize}
		\item $a =b$. In this case we have have $x = a \mem a$.
			From the definition of equality $a \mem a = 0 \mem 0$. 
			From identifying natural numbers with integers $n \mem 0$
			we have $x = 0$.
		\item $a > b$. In this case we have some $c\in \N$ such 
			that $a = b + c$. Thus we have $x = (b+c) \mem b$. Again 
			by equality of integers we have $x = c \mem 0 = c$, 
			which is a positive natural number.
		\item $a < b$. In this case we have some $c \in \N$ such 
			that $b = a + c$. Thus we have $x = a \mem (a+c)$.
			This gives us $x = 0 \mem c = -(c \mem 0) = -c$ by 
			the definition of negation.
	\end{itemize}
	We leave showing that only 1 can happen at a time as an exercise
	to the reader (basically just look at every possible pair
	happening and show that they are impossible). \hfill $\blacksquare$
\end{lemma}

If $n$ is a positive natural number, we call $-n$ a negative integer. We
now summarize the algebraic properties of the integers.

\begin{proposition}
	(Laws of algebra for integers). Let $x,y,z$ be integers. Then 
	we have:
	\begin{align*}
		x + y &= y + x	
		(x+y) + z &= x + (y+z) \\
		x + 0 = 0 + x &= x \\ 
		x + (-x) = (-x) + x &= 0 \\ 
		xy = yx \\ 
		(xy)z &= x(yz) \\ 
		x1 = 1x &= x \\ 
		x(y+z) &= xy + xz \\ 
		(y+z)x &= yx + zx
	\end{align*}

	\it{Proof}. The easiest way to prove this to to define:
	$$
	x = (a \mem b) \quad y = (c \mem d) \quad z = (e \mem f)
	$$
	then write out each identity in terms of the above, expand 
	using the algebra of integers we already have and then 
	the algebra of natural numbers. For example:
	\begin{align*}
		(xy)z &= (a\mem b)(c\mem d)(e\mem f) \\	
		      &= ((ac + bd) \mem (ad + bc)) (e \mem f) \\
		      &= ((ace + bde + adf + bcf) \mem (acf + bdf + ade + bce)) \\
		x(yz) &= (a\mem b)((ce + df) \mem (cf + de)) \\
		      &= (ace + bde + adf + bcf) \mem (acf + bdf + ade + bce)
	\end{align*}
	We leave proving the rest as an exercise to the reader. \hfill $\blacksquare$
\end{proposition}

\begin{remark}
	The above nine identities assert that the intergers form 
	a \it{commutative ring}. This means the set integers forms 
	a commutative additive group, with an additional 
	multiplication operation that is associative, commutative,
	distributive over addition.
	If $xy \neq yx$ then it would just be a ring. If the 
	set had multiplicative inverses, then it would also 
	form a multiplicative commutative group, and thus 
	would be a \it{field}. The rational $\Q$ will be
	the first field that we encounter.
\end{remark}

\begin{definition}
	(Subtraction). We define the operation of subtraction to be: 
	$$
	x - y := x + (-y)
	$$
\end{definition}

We do not need to verify the substiution axiom for this opertaion,
since it is defined in terms of two operations on integers 
(addition and negation) that already obey this axiom.

Now lets get rid of the pesky $\mem$ symbol! Let $a,b \in \N$. Then 
we have: 
$$
a - b = a + (-b) = (a \mem 0) + (0 \mem b) = (a \mem b)
$$
We now generalize a couple of propositions that we had for the natural 
numbers to the integers.
\begin{proposition}
	(Integers have no 0 divisors). For $a,b \in \Z$ such 
	that $ab = 0$, then either $a = 0$ or $b = 0$.
\end{proposition}

\begin{proposition}
	(Cancellation law of integers). Let $a,b,c \in \Z$ such that
	$ac = bc$ and $c \neq 0$. Then $a = b$.
\end{proposition}

We now repeat the definition of order (that is defining $<$ and $>$) 
verbatum for the integers.

\begin{definition}
	(Ordering of the integers). Let $n$ and $m$ be integers. 
	$n \geq m$ iff $n = m + a$ for some natural number $a$.
	$n > m$ iff $n \geq m$ and $n \neq m$. 
\end{definition}

\begin{lemma}
	(Properties of order of integers). We summarize some simple 
	facts. Let $a,b,c \in \Z$, then:
	\begin{enumerate}
		\item $a > b \iff a -b$ is a positive natural number.
		\item (Addition preserves order). $a > b \implies a + c > b + c$.
		\item (Positive ultiplication preserves order). 
			$a > b \land c > 0 \implies ac > bc$.
		\item  (Negative multiplication reverses order). 
			$a > b \land c < 0 \implies ac < bc$.
		\item (Order is transitive). $a > b \land b > c \implies a > c$.
		\item (Order is trichotomous). Exactly one of $a > b$, $a = b$, 
			or $a < b$ is true.
	\end{enumerate}
\end{lemma}

\exercisesline

\it{Exercise 4.1.4}. Show that $(-1) \times a = -a$ for any integer $a$.

\it{Proof}. For $a \in \Z$ we have $a = n \mem m$ for $n,m \in \N$ 
and $-1 = 0\mem 1$. We get: 
\begin{align*}
	(-1) \times a &= (0 \mem 1)(n \mem m) \\
		      &= (0n + 1m) \mem (0m + 1n)  \\
		      &= (m \mem n) = -a
\end{align*}
from the definition of negation. \hfill $\blacksquare$

\it{Exercise 4.1.8}. Show the the principle of induction does 
not apply to the integers. That is exhibit a property $P(x)$ for 
$x \in \Z$ such that $P(0)$ is true, and $P(x) \implies P(x+1)$, 
but $P(n)$ is not true $\forall n \in \Z$.

\it{Proof}. Consider an integer $x$ that is $m - n$ for 
natural numbers $m,n$. Let $P(x) = P(m-n)$ be the property that $m \geq n$. 
We have $P(0) = P(0-0)$ is true as $0 \geq 0$. If $P(x)$ is true 
then $P(x+1)$ is true as $x+1 = (m+1) - n$ and $m+1 \geq n$ as $m \geq n$.
Note that for negative integers, $P(x)$ is untrue however.
\hfill $\blacksquare$

\subsection{Rational Numbers}

We defined the integers with addition, multiplication, subtraction, 
and order and verified all the algebraic and order-theoretic 
properties. We now add build the rationals, adding division 
to our list of operations.

As the integers were constructed by subtracting two natural 
numbers, the rations are constructed by dividing two integers.

We know what we expect, that $a/b = c/d$ iff $ad = bc$. Just 
like we did with the integers, we create a new meaningless 
symbol $//$ which will eventually be replaced with the devision 
symbol, and make the following definition.

\begin{definition}
	(Rational numbers). A rational number is an expression of the	
	form $a//b$, where $a$ and $b$ are integers and $b \neq 0$.
	Two rational numbers $a//b$ and $c//d$ are equal, $a//b = c//d$,
	iff $ad = bc$. The set of rational numbers is denoted by $\Q$.
\end{definition}

For full rigour we should show this is a valid definition of equality
by showing it is reflexive, symmetric, and transitive, and obeys
the axiom of substitution. We will not do this here. Now we 
need to define addition, multiplication, and division, which 
follow our intution that $a/b + c/d = (ad + bc)/(bd)$,
and $a/b \times c/d = ac/bd$ and $-(a/b) = (-a)/b$.

\begin{definition}
	(Addition, multiplication, and negation of rationals). 
	Let $a,b,c,d \in \Z$ with $b,d \neq 0$. We define:
	\begin{align*}
		(a//b) + (c//d) &= (ad + bc)//(bd) \\
		(a//b) \times (c//d) &= (ac)//(bd) \\
		-(a//b) &= (-a)//b
	\end{align*}
\end{definition}

Note that if $b$ and $d$ are non zero, then $bd$ is non zero, so addition 
and multiplication are closed over the rationals. 
\begin{lemma}
	Addition, product, and negation are well defined on the integers.
	This means that if one replaces $a//b$ with $a'//b'$ with 
	$a//b = a'//b'$, then the output of the operations remains 
	the same, and the same is true for $c//d$.
\end{lemma}

\begin{remark}
	At this point when we are in abstract land, why do we 
	not allow dividing by $0$? This is because if we did,
	then $(a/0) \times (0/1) = (a0/0) = (a/1)$ by the definition
	of rational number equality (as $a0 = a0$), but it 
	would also equal $(0/0) = 0$, which is a contradiction 
	if $a \neq 0$.
\end{remark}

\begin{remark}
	The rational numbers $a //1$ behave identically to the integer $a$:

	\begin{align*}
		(a // 1) + (b // 1) &= (a + b)//1 \\	
		(a // 1) \times (b // 1) &= (ab)//1 \\ 
		-(a // 1) &= (-a)//1
		a//1  = b//1 \iff a = b
	\end{align*}

	Because of this, we will identify $a$ with $a//1$ for all integers 
	$a$.
\end{remark}

We define the reciprocal operation on the rations, which is 
analogous to the negation operation on the integers.

\begin{definition}
	(Reciprocal) For non-zero $x = a//b$ we define the reciprocal
	to be $x^{-1} = b//a$. This preserves equaity (axiom of 
	substitution).
\end{definition}

Note that an operation such as ``numerator'' does note respect
the axiom of substitution, so we cannot include it. This 
means we need to be careful in proofs when we say things like
``the nubmerator of $x$ is $a$'' and then use this fact. We 
also have that the reciprocal of $0$ is undefined. 

\begin{proposition}
	(Laws of algebra on the rationals). Let $x,y,z$ be 
	rational numbers, then we have:
	\begin{align*}
		x + y &= y + x \\
		(x+y) + z &= x + (y+z) \\
		x + 0 = 0 + x &= x \\
		x + (-x) = (-x) + x &= 0 \\
		xy = yx \\
		(xy)z &= x(yz) \\
		x1 = 1x &= x \\
		x(y+z) &= xy + xz \\
		(y+z)x &= yx + zx	
	\end{align*}

	If $x$ is non-zero, then: 
	$$
	xx^{-1} = x^{-1}x = 1
	$$

	\it{Proof}. The proof is long and invovled, but similar 
	to proving the algebraic properties of the integers,
	we simply write $x = a//b$, $y = c//d$, etc. and 
	verify each identity in turn. We leave this as 
	an exercise to the reader. \hfill $\blacksquare$
\end{proposition}

\begin{remark}
	Note that the above algebraic properties match that of
	the integers exactly except for the the additional 
	of the final idnetity involving reicprocals. This identity,
	which states the existance of multiplicative inverses for 
	all elements of the set except for the addititive identity $0$, 
	makes the rationals a \it{field}. This is the first field
	we have encountered (recall that the integers were a 
	commtuative ring).
\end{remark}

\begin{definition}
	(Quotient) The quotient of rational numbers $x$ and $y$, 
	provided that $t$ is non-zero, is defined as: 
	$$
	x / y = x \times y^{-1}
	$$
\end{definition}

For example $(3//4)/(5//6) = (3//4) \times (6//5) = 18//20 = 9//10$. 
Using this definition, we can see that $a/b = a//b$ for every 
integer $a$ and non-zero integer $b$. This is because:
$$
a/b = a \times b^{-1}= (a//1) \times (b//1)^{-1} = a//1 \times 1//b = a//b
$$
Thus we can discard $//$  and simply use $/$.

\begin{definition}
	(Subtraction on the rations). We define subtraction on the rationals 
	identically as we did for the integers:
	$$
	x - y = x+ (-y)
	$$
\end{definition}

\begin{definition}
	A rational number $x$ is positive iff $x = a/b$ for some 
	positive integers $a,b$. It is egative iff $x = -y$ for 
	some positive rational number $y$. 
\end{definition}

\begin{lemma}
	(Trichotemy of rationals). If $x$ is ration, then it is 
	positive, negative, or 0.
\end{lemma}

\begin{lemma}
	(Ordering of rationals). If $x,y$ are rational, $x > y$ 
	iff $x-y$ is a positive rational, and $x < y$ 
	is a negative rational. We write $x \geq y$ iff either 
	$x > y$ or $x =y$.
\end{lemma}

\begin{proposition}
	(Propterites of order on the rationals). We've seen these 
	all before for the integers:
	\begin{enumerate}[label=(\alph*)]
		\item (Order trichotemy). One of $x > y$, $x = y$, or 
			$x < y$ is true.
		\item (Order is anti-symmetric). $x < y$ iff $y > x$.
		\item (Order is transitive). $x > y$ and $y > z$ implies 
			$x > z$.
		\item (Addition preserves order). $x > y$ implies $x + z > y + z$.
		\item (Positive mult preserves order). $x > y$ and $z > 0$ implies 
			$xz > yz$.
	\end{enumerate}
\end{proposition}

The above properties combined with the field algebraic properties combine
to make $\Q$ an \it{ordered field}. 

\exercisesline

\it{Exercise 4.2.6}. Show that if $x,y,z\in\Q$ such that $x < y$ and 
$z$ is negative, then $xz > yz$.

\it{Proof}. We have $x < y$ so $x - y$ is positive. We have $z$ is
negative so $-z$ is positive. Thus $x-y$ and $-z$ are positive so:
\begin{align*}
	x &< y \\ 
	x (-z) &< y (-z) \\ 
	-xz  &< -yz \\ 
	-xy + xz &< -yz + xz \\ 
	0 &< xz - yz \\ 
	yz &< xz \\
	xz &> yz 
\end{align*}


\subsection{Absolute Value and exponentiation} 

We have defined addition, multiplication, subtraction, and division
on the rationals, with the latter two being defined in terms of the
more primitve negations $x + (-y)$  and reciprocal $x \times y^{-1}$ 
operations. We can now define other operations. Here we introduce 
absolute value and exponentiation.

\begin{definition}
	(Absolute value). If $x$ is rational, then the absolute value 
	$|x|$ is defined as:
	$$
	|x| = \begin{cases}
		x & \text{if $x$ is positive}  \\	
		-x & \text{if $x$ is negative} \\ 
		0 & \text{if $x = 0$}
	\end{cases}
	$$
\end{definition}

\begin{definition}
	(Distance). Let $x,y\in\Q$. The quantity $|x-y|$ os ca;;ed 
	the distance between $x$ and $y$, sometimes denoted $d(x,y)$.
\end{definition}

\begin{proposition}
	(Basic properties of absolute value and distance). For 
	$x,y,z \in Q$ we have: 
	\begin{enumerate}[label=(\alph*)]
		\item (Non degenerate). $|x| \geq 0$. Also $|x| = 0$ iff 
			$x =0$. 
		\item (Traingle inequality for abs). $|x + y| \leq |x| + |y|$.
		\item $-y \leq x \leq y$ iff $|x| \leq y$. Thus we have 
			$-|x| \leq x \leq |x|$. 
		\item (Multiplicity of abs). $|xy| = |x||y|$. Thus $|-x|=|x|$.
		\item (Non gen of distance). $d(x,y) \geq 0$. Also $d(x,y) = 0$ 
			iff $x = y$.
		\item (Symmetry of distance). $d(x,y) = d(y,x)$.
		\item (Triangle inequality for distance).
			$d(x,z) \leq d(x,y) + d(y,z)$.
	\end{enumerate}
\end{proposition}

Distance (and thus absolute value) are useful for measuring
how close two numebrs are. 

\begin{definition}
	($\vep$-closeness). Let $\vep > 0$ be a rational number, and 
	$x,y$ be rational. We say that $y$ is $\vep$-close to $x$ 
	iff $d(x,y) < \vep$.
\end{definition}

Note that this definition is not standard in other textbooks. We 
use it to build scaffolding and then discard it later, much like 
$//$ and $\mem$ for the	rationals and integers respectively.

\begin{proposition}
	(Properties of $\vep$-closeness). Consider $x,y,z,w\in\Q$. We have:
	\begin{enumerate}[label=(\alph*)]
		\item  If $x=y$, then $x$ is $\vep$-close to $y$ for 
			any $\vep > 0$. Conversely, if $x$ is 
			$\vep$-close to $y$ for all $\vep > 0$, then
			$x=y$.
		\item (Symmetric). If $x$ is $\vep$-close to $y$, then 
			$y$ is $\vep$-close
		\item (Sort of transititve). If $x$ and $y$ 
			are $\vep$ close and $y$ and $z$ are $\delta$ 
			close then $x$ and $z$ are $(\vep + \delta)$ close.
		\item Let $\vep, \delta > 0$. If $x$ and $y$ are
			$\vep$ close and $z$ and $w$ are $\delta$ close,
			then $x+ z$ and $y +w$ are $(\vep + \delta)$ close.
			Also $x -z$ and $y-w$ are $(\vep + \delta)$ close.
		\item If $x$ and $y$ are $\vep$ close, then they
			are $\vep'$ close $\forall \vep' > \vep$.
		\item If $y$ and $z$ are both $\vep$ close to $x$, and 
			$y \leq w \leq z$ or $z \leq w \leq y$, then 
			$w$ is $\vep$ close to $x$.
		\item If $x$ and $y$ are $\vep$ close, and $z \neq 0$, then 
			$xz$ and $yz$ are $\vep|z|$ close.
		\item If $x$ and $y$ are $\vep$ close, and $z$ and $w$ 
			are $\delta$ close, then $xy$ and $yw$ are 
			$(\vep |z| + \delta |y| + \vep \delta)$ close.
	\end{enumerate}

	\it{Proof}. We prove only (h). We want to say something 
	about $|yw - xz|$ We will write $yw$ in terms of $x$ and $z$.
	
	Let $a = x-y$, thus $y = x +a$ with $|a| \leq \vep$. Let $b = w-z$,
	thus $w = z + b$ with $|b| \leq \delta$. We have:
	$$
	yw = (x+a)(z+b) = xz + az + xb + ab
	$$
	This gives us:
	\begin{align*}
		|yw - xz| &= |az + xb +ab | \\ 
			  &\leq |az| + |xb| + |ab|  \tag*{By triangle ineq}\\
			  &\leq |z||a| + |x||b| + |a||b| \tag*{By multiplicity} \\
			  &\leq \vep|z| + \delta|x| + \vep\delta
	\end{align*}

	Thus we have $yw$ and $xz$ are $(\vep |z| + \delta |y| + \vep \delta)$
	close. \hfill $\blacksquare$
\end{proposition}

Now we define exponentiation for natural numbers recursively as we
did with multiplicaiton. 

\begin{definition}
	(Exponentiation of a natural number). Let $x$ be a 
	rational number. To raise $x$ to the pwer $0$ 
	we define $x^{0} :=1$, and thus $0^0=1$. Now suppose 
	inductively that we have defined $x^{n}$ for some
	$n \in \N$. We define $x^{n+1} := x^{n} \times x$.
\end{definition}

We now do so for negative integers.

\begin{definition}
	(Exponentation to a negative numebr). Let $x$ be a non-zero 
	rational number. Then for any negative integer
	$-n$, we define $x^{-n} := 1/x^n$.
\end{definition}

We now have $x^n$ defined for any integer $n$, and it is closed 
over the rationals. 

\begin{proposition}
	(Properties of exponentiation). For $x,y$ non-zero 
	rational numbers, we have:
	\begin{enumerate}[label=(\alph*)]
		\item $x^n x^m = x^{n+m}$. $(x^n)^m = x^{nm}$. $(xy)^n = x^n y^n$.
		\item For $n>0$, $x^n =0$ iff $x=0$.
		\item If $x \geq y > 0$ then $x^n \geq y^n > 0$ if $n$ is positive.
			and $0 < x^n \leq y^n$ if $n$ is negative.
		\item If $x,y > 0$, $n \neq 0$, and $x^n = y^n$, then $x=y$.
		\item We have $|x^n| = |x|^n$.
	\end{enumerate}
\end{proposition}

\exercisesline

\it{Exercise 4.3.5}. Prove that $2^N \geq N$ for all positive integers $N$. 

\it{Proof}. We do so by induction. For $N=1$ we have $2^1 = 2 \geq 1$. 
Assume truen for $N=n$. Now consider $N=n+1$, we have:
\begin{align*}
	2^{n+1} &= 2^n \times 2 \\ 
		&\geq N \times 2 \tag*{By inductive assumption}\\ 
		&= N + N \\
		&\geq N + 1 \tag*{As $N \geq 1$}
\end{align*}

\subsection{Gaps in the rational numbers}

This is a non-rigerous argument, but consider lining up all 
the rational numbers on a line from $y$ to $x$ (for $y < x$). 
Inside the rationals we have the integers. 

\begin{proposition}
	(Interspersing of integers by rationals). Consider $x\in\Q$. 
	There exists an $n\in\Z$ such that $n \leq x < n+1$. Thus 
	there exists and $N \in \N$ such that $N > x$. Thus there 
	is no such thing as a rational number which is larger 
	than all naturals.
\end{proposition}

\begin{remark}
	In integer $n$  for which $n \leq x < n+1$ is sometimes 
	called the integer part, and is $n = \lfloor x \rfloor$.
\end{remark}

\begin{proposition}
	(Interspersing of rationals by rationals). For $x,y \in \Q$ 
	such that $x < y$, there exists $z \in \Q$ such that
	$x < z < y$.

	\it{Proof}. Start with $z = (x+y)/2$. Sinze $x < y$ 
	and $1/2 = 1//2$ is positive, we have $x/2 < y/2$. Adding 
	$y/2$ to both sides we get $z < y$. Do same for $x_2$ and 
	we conclude the proof. \hfill $\blacksquare$
\end{proposition}

Despite the rationals having this denseness to them, there are 
still ``gaps'' between rationals. The denseness does ensure these
``gaps'' are infinitesimally small, but they are still there.

\begin{proposition}
	There does not exists a rational $x$ such that $x^2 = 2$.

	\it{Proof}. Assume such an $x$ exists. We can assume 
	it is positive (if it were not, then replace with $-x$ 
	as $(-x)^2 = x^2$. So for $p,q \in \N$ we have: 
	$$
	x = p/q \implies x^2  = p^2/q^2 = 2 \implies p^2 = 2q^2
	$$
	Thus we have that $p^2$ is even. Thus $p$ is even, as 
	otherwise $p^2$ would be odd. So we have $p = 2k$. Thus 
	$2q^2 = 4k^2 \implies q^2 = 2k^2$. Thus $q$ is even, 
	and $q = 2l$ for some $l$. Note that $k < p$ and $ l < q$, 
	and all are natural numbers. We can repeat this 
	process infinitely, which contradicts the 
	principle of infinite descent.
\end{proposition}

We can however get rationals that are arbitrarily close to root 2.

\begin{proposition}
	For every rational $\vep >0$, there exists 
	a non-negtaive rational $x$ such that $x^2 < 2 < (x + \vep)^2$. 
\end{proposition}

What this means is that we can get as close as we want to $\sqrt{2}$.
For example:

$$
1.4, 1.41, 1.414, 1.4142, 1.41421, \ldots
$$

(here and going forward we use terminating decimals which can 
simply be written as rationals). From the above, it seems 
like we can make root 2 by taking the ``limit'' of 
a sequence of rational numbers. This is how we will 
construct the real numebrs in the next section.\footnote{There
are other ways to make the reals, in particular using 
``Dedekind cuts'', or using infinite decimal expansions.} 

\exercisesline

\it{Exercise 4.4.2}. A sequence $a_0, a_1, a_2, \dots$ of numbers 
(natural, interger, rational, real) is said to be in \it{infinite descent}
if we have $a_n > a_{n+1}$ for all natural numbers $n$. Prove 
the \it{principle of infinite descent}: that it is not possible 
to have a sequence of natural numbers which is in infinite descent.

\it{Proof}. Assume towards contradiction that such a sequence 
did exist. As all $a_i$ are natural numbers we have $a_i \geq 0$. 
We use induction to show that $a_i \geq k$ for all $k$. Assume 
that $a_i \geq n$. Now consider $n+1$. Assume towards contradiction 
there is some $a_i < n+1$. Then $a_{i+1} < n$, which is a contradiction 
as all the values are greater than $n$. Thus we have $a_i \geq n+1$
for all $i$. Now pick $k = a_1$, and we have that $a_2 > k = a_1$ and
thus the sequence is not in infinite descent, which is
a contradiction. \hfill $\blacksquare$




%%%%%%%%%%%%%%%%%%%%%
% Section 5         %
%%%%%%%%%%%%%%%%%%%%%

\section{Real Numbers}


Recap of what we have done so far:
\begin{enumerate}
	\item Defined natural numbers using the Peano axioms, and 
		postulated that such a number system exists. Using 
		the axioms, we recursively defined addition 
		and multiplication and showed they obeyed our 
		concepts of algebra on the naturals.
	\item We constructed the integers using the notion 
		of difference between two natural numbers $a\mem b$. 
	\item We constructed the rationals using the notion of 
		quotient between two integers $a//b$, but excluded
		dividing by $0$ to keep the laws of algebra consistent.
\end{enumerate}

The rationals are useful, but fail in places like geometry
and trigonometry. We must thus replace the rational number 
line with the real number line. We will also need real 
number for calculus.

We need more machinary to construct the reals than just 
aiming to add a new operation (like what we did for the integers 
and rationals). In particular, we need to define 
a limit. 

The real numbers will be similar to the rationals, but with 
some new operations, in particular \it{supremum}, that we then 
use to define limits. When we give the procedure of constructing
the reals using limits of sequences of rationals, this is an 
example of a broader concept known as \it{completing} one 
metric space from another.

\subsection{Cauch Sequences}

\begin{definition}
	(Sequences). $m\in\Z$. A sequence $(a_n)_{n=m}^{\infty}$ of 
	rational numbers is any function $f: \{n \in Z: n \geq m \} \to \Q$.
	I.e., a mapping that assigns each integer $n$  greater than or
	equal to $m$ a rational number $a_n$. More informally, it is 
	simply a collection of rationals.
\end{definition}

We want to define reals as limits of sequences of rationals. To 
do this we need to distinguish what sequences converge and what do not.

\begin{definition}
	($\vep$-steadiness). Let $\vep > 0$. A sequence $(a_n)_{n=0}^\infty$ 
	is said to be $\vep$-steady iff each pair $a_j, a_k$ of elements 
	is $\epsilon$-close for every natual number $j,k$. 

	In other words, the sequence $a_0, a_1, a_2, \dots$ is $\vep$-steady
	iff $d(a_j,a_k) < \vep$ for all $j,k \in \N$.
\end{definition}

\begin{remark}
	The above definition is not standard in the literature and 
	is just used for scaffolding in this section. The same 
	for the below.
\end{remark}

\begin{definition}
	(Eventual $\vep$-steadiness). A sequence $(a_n)_{n=0}^\infty$	
	is said to be eventually $\vep$-steady iff there exists
	an $N \geq 0$ such that $a_N, a_{N+1}, a_{N+2}, \dots$ is 
	$\vep$-steady.

	In other words, if $\exists N \geq 0$ such that 
	$d(a_j, a_k) < \vep$ for all $j,k \geq N$.
\end{definition}

\begin{example}
	The sequence $a_n = 1/n$ is not $0.1$-steady but is $0.1$-eventual 
	steady.	The sequence $10, 0, 0,0, \dots$ is not $\vep$-steady 
	for any $\vep < 10$, but is $\vep$-eventual steady for any
	$\vep > 0$ (not this is strict as the definition of $\vep$ 
	closeness uses a strict less than inequality, so nothign 
	can be $0$-close.
\end{example}

We now define the notion of what it means for a sequnce to ``want'' 
to converge (this doesn't mean it will).

\begin{definition}
	(Cauchy sequence). A sequence $(a_n)_{n=0}^\infty$ of rational
	numbers is a Cauchy sequence iff $\forall \vep > 0$, the
	sequnce is eventually $\vep$-steady.

	That is the sequnce is Cauchy iff $\forall \vep > 0$, there
	exists and $N\geq 0$ such that $d(a_j, a_k) < \vep$ for all
	$j,k \geq N$.
\end{definition}

\begin{remark}
	So far we have $\vep$ is rational as we have not defined 
	the reals. Once we have the reals, we will change 
	the definition to allow $\vep$ to be real, and show 
	that:
	\begin{center}
		Sequnce is $\vep$-eventuallly steady $\forall \vep > 0$,
		$\vep \in \Q$ $\iff$ \\
		Sequnce is $\vep$-eventuallly steady $\forall \vep > 0$,
		$\vep \in \R$.
	\end{center}
\end{remark}

\begin{proposition}
	The sequnce $a_n := 1/n$ is Caucy.

	\it{Proof}. Let $\vep > 0$. We want to show that the sequence
	is eventually $\vep$-steady. We have $\vep = a/b$ for positive 
	$a,b$. There must exists some $N \in \N$ such that $\vep > 1/N$.
	For all $n,m > 2N$ we have $1/n, 1/m < /2N$. Thus:
	\begin{align*}
		|1/n - 1/m| &\leq |1/n| + |1/m| \\
			    &< 1/2N + 1/2N = 
			    &=1/N \\
			    &< \vep
	\end{align*}
	Which concludes the proof. \hfill $\blacksquare$

	Note that we know such an $N$ exists because for every rational sits 
	between two consequtive reals.
\end{proposition}

\begin{definition}
	(Bounded sequence). Let $M \geq 0$ be rational. A sequence 
	$a_1, \dots, a_n$
	is bounded by $M$ iff $|a_i| \leq M$ for all $1 \leq i \leq n$.

	An infinite sequence is bounded iff $|a_i| \leq M$ for all $i$.

	A sequence is bounded iff there exists a rational $M \geq 0$ such that
	the sequence is bounded by $M$.
\end{definition}

\begin{lemma}
	Every finite sequence $a_1, \dots, a_n$ is bounded.	

	\it{Proof} (brief). Induct on $n$. If $n=1$ then 
	bounded by $|a_1|$. Assume true for $n$. Consider $n+1$.
	$a_1, \dots, a_{n+1}$ is bounded by $M + |a_{n+1}|$.
	\hfill $\blacksquare$
\end{lemma}

\exercisesline

\it{Exercise 5.1.1}. Show that every Cauchy sequence 
$(a_n)_{n=0}^\infty$ is bounded. 

\it{Proof}. As it is Cauchy, there exists an $N \in \N$ such that 
for all $j,k  \geq N$ we have $|a_j - a_k| < 1$. Now 
consider $a_n, \dots, a_{N-1}$. This is finite so by the above 
lemma is bounded. Whats more $a_{N}, a_{N+1}, \dots$ is 
bounded as it is $1$-steady. Thus the entire sequence is
bounded. \hfill $\blacksquare$


\subsection{Eequivalence Cauchy Sequences}

Consider two equences:
$$
1.4, 1.41, 1.414, 1.4142, 1.41421, \ldots
$$
and 
$$
1.5, 1.42, 1.415, 1.4143, 1.41422, \ldots
$$

Inforammly it seems these two sequences are both converging 
to $\sqrt{2}$.  We want to define reals as the limits 
of Caucy sequences, so we need to know when two sequences 
give the same limit, but that is circular because a limit 
will be a real number, which we have not introduced yet.

So we need some other definition to say ``these two sequences 
are the same'' or ``these two sequences are similar.''

\end{document}




